\documentclass[12pt]{article}
\usepackage[utf8]{inputenc}
\usepackage[russian]{babel}
\usepackage{amsmath,amssymb}
\usepackage{graphics}
\usepackage{pbox}
\usepackage[x11names]{xcolor}
\definecolor{brightmaroon}{rgb}{0.76, 0.13, 0.28}
\definecolor{royalazure}{rgb}{0.0, 0.22, 0.66}
\usepackage[colorlinks=true,linkcolor=royalazure]{hyperref}
\usepackage{tikz, tkz-fct, pgfplots}
\usetikzlibrary{arrows}
\usepackage{geometry}
\geometry{
	a4paper,
	total={170mm,257mm},
	left=20mm,
	top=20mm
} 
\usepackage[labelsep=period]{caption}

% ----------------- Commands ----------------- 
\newcommand{\eps}{\varepsilon}
\newcommand\tline[2]{$\underset{\text{#1}}{\text{\underline{\hspace{#2}}}}$}

% ----------------- Set graphics path ----------------- 
\graphicspath{{img/}}

\begin{document}
\newpage 
\pagestyle{empty}
\centerline{\large Министерство науки и высшего образования}	
\centerline{\large Федеральное государственное бюджетное образовательное}
\centerline{\large учреждение высшего образования}
\centerline{\large ``Московский государственный технический университет}
\centerline{\large имени Н.Э. Баумана}
\centerline{\large (национальный исследовательский университет)''}
\centerline{\large (МГТУ им. Н.Э. Баумана)}
\hrule
\vspace{0.5cm}
\begin{figure}[h]
\center
\includegraphics[height=0.35\linewidth]{bmstu-logo-small.png}
\end{figure}
\begin{center}
	\large	
	\begin{tabular}{c}
		Факультет ``Фундаментальные науки'' \\
		Кафедра ``Высшая математика''		
	\end{tabular}
\end{center}
\vspace{0.5cm}
\begin{center}
	\LARGE \bf	
	\begin{tabular}{c}
		\textsc{Отчёт} \\
		по учебной практике \\
		за 1 семестр 2020---2021 гг.
	\end{tabular}
\end{center}
\vspace{0.5cm}
\begin{center}
	\large
	\begin{tabular}{p{5.3cm}ll}
		\pbox{5.45cm}{
			Руководитель практики,\\
			ст. преп. кафедры ФН1} 	& \tline{\it(подпись)}{5cm} & Кравченко О.В. \\[0.5cm]
		студент группы ФН1--11 		& \tline{\it(подпись)}{5cm} & Кузнецова Е.Д.
	\end{tabular}
\end{center}
\vfill
\begin{center}
	\large	
	\begin{tabular}{c}
		Москва, \\
		2020 г.
	\end{tabular}
\end{center}
\newpage	
\tableofcontents
\newpage
\section{Цели и задачи практики}	
\subsection{Цели}
--- развитие компетенций, способствующих успешному освоению материала бакалавриата и необходимых в будущей профессиональной деятельности.
\subsection{Задачи}
\begin{enumerate}
\item Знакомство с программными средствами, необходимыми в будущей профессиональной деятельности.
\item Развитие умения поиска необходимой информации в специальной литературе и других источниках.
\item Развитие навыков составления отчётов и презентации результатов.
\end{enumerate}
\subsection{Индивидуальное задание}	
\begin{enumerate}
\item Изучить способы отображения математической информации в системе вёртски \LaTeX.
\item Изучить возможности  системы контроля версий \textsf{Git}.
\item Научиться верстать математические тексты, содержащие формулы и графики в системе \LaTeX.
Для этого, выполнить установку свободно распространяемого дистрибутива \textsf{TeXLive} и оболочки \textsf{TeXStudio}.
\item Оформить в системе \LaTeX типовые расчёты по курсе математического анализа согласно своему варианту.
\item Создать аккаунт на онлайн ресурсе \textsf{GitHub} и загрузить исходные \textsf{tex}--файлы 
и результат компиляции в формате \textsf{pdf}.
\end{enumerate} 
\newpage
\section{Отчёт}
Актуальность темы продиктована необходимостью владеть системой вёрстки \LaTeX и средой вёрстки \textsf{TeXStudio} для
отображения текста, формул и графиков. Полученные в ходе практики навыки могут быть применены при написании
курсовых проектов и дипломной работы, а также в дальнейшей профессиональной деятельности.
Ситема вёрстки \LaTeX содержит большое количество инструментов (пакетов), упрощающих отображение информации в различных 
сферах инженерной и научной деятельности. 
% ---------------------------------------------------
\newpage
\section{Индивидуальное задание}
\subsection{Пределы и непрерывность.}
\begin{center}
\textbf{Задача № 1.}   
\end{center}
\textbf{Условие:}
Дана последовательность  $a_{n}=\dfrac{4n-3}{2n+1}$ и число $c=2$. Доказать,что $\lim\limits_{x\rightarrow\infty} a_{n}=c$, а именно для кажого $\eps > 0$ найти наименьшее натуральное число $N = N(\eps)$ такое, что $|a_{n}-c|<\eps$. Заполнить таблицу:
\begin{table}[h]
    \centering
    \begin{tabular}{|c|c|c|c|}
        \hline
         $\eps$ & $0,1$ & $0,01$ & $0,001$ \\
         \hline
         $N(\eps)$ &  &  & \\
         \hline
    \end{tabular}
\end{table}
\textbf{Решение:}
$$a_{n}=\dfrac{4n-3}{2n+1}; \; c = 2.$$ 
Найдём предел $a_{n}$:
$$\lim\limits_{x\rightarrow\infty} a_{n}= 2 = c.$$
Рассмотрим  $|a_{n}-c|<\eps$:
$$\biggl |\dfrac{4n-3}{2n+1} - 2  \biggr |<\eps,$$
$$\biggl |\dfrac{4n-3-4n-2}{2n+1} \biggr | <\eps,$$
$$\dfrac{5}{2n+1} < \eps,$$
$$n > \dfrac{5-\eps}{2\eps}.$$
При $\eps = 0,1$ получим:
$$ n > \dfrac{5-0,1}{2*0,1} \Leftrightarrow n > 24,5.$$
При $\eps = 0,01$ получим:
$$ n > \dfrac{5-0,01}{2*0,01} \Leftrightarrow n > 249,5.$$
При $\eps = 0,001$ получим:
$$ n > \dfrac{5-0,001}{2*0,001} \Leftrightarrow n > 2499,5.$$
Заполним таблицу:
\begin{table}[h]
    \centering
    \begin{tabular}{|c|c|c|c|}
        \hline
         $\eps$ & $0,1$ & $0,01$ & $0,001$ \\
         \hline
         $N(\eps)$ & $25$ & $250$  & $2500$ \\
         \hline
    \end{tabular}
\end{table}
\newpage
\begin{center}
\textbf{Задача № 2.}   
\end{center}
\textbf{Условие:}
Вычислить пределы функций
\begin{table}[h]
    \centering
    \begin{tabular}{|c|c|}
        \hline
         a & $\lim\limits_{x\rightarrow 2} \dfrac{x^3-2x^2+x-2}{x^3+x^2-8x+4}$ \\
         \hline
         б & $\lim\limits_{x\rightarrow+\infty} \dfrac{2+\sqrt[3]{1-8x^4}}  {1-3\sqrt[5]{x^6}}$ \\
         \hline
         в & $\lim\limits_{x\rightarrow 0} \dfrac{\sqrt[3]{27+x}-\sqrt[3]{27-x}}{x+2\sqrt[3]{x^4}}$\\
         \hline
         г & $\lim\limits_{x\rightarrow 0} \biggl(5-\dfrac{4}{\cos{2x}}\biggl)^{\frac{1}{\sin^2{x}}}  $\\
         \hline
         д & $\lim\limits_{x\rightarrow 0} \biggl(\dfrac{\exp{x}-1}{\arctg2x}\bigg)^\frac{6x}{\lg({1+x})}$ \\
         \hline
         е & $\lim\limits_{x\rightarrow 1} \dfrac{1+\cos{\pi x}}{\tg^2(\pi x)}$\\
         \hline
    \end{tabular}
\end{table}
\textbf{Решение:}\\
а)
$$\lim\limits_{x\rightarrow 2} \dfrac{x^3-2x^2+x-2}{x^3+x^2-8x+4}.$$
При подстановке $ x=2$ в числитель и знаменатель получаем неопределённость вида $\biggl[\dfrac{0}{0}\biggr].$
Разложим на множители:
$$\lim\limits_{x\rightarrow 2} \dfrac{(x-2)(x^2+1)}{(x-2)(x^2+3x-2)} .$$
Сократим одинаковые множители:
$$\lim\limits_{x\rightarrow 2} \dfrac{x^2+1}{(x^2+3x-2)}=\dfrac{5}{8}.$$
б)
$$\lim\limits_{x\rightarrow+\infty}\dfrac{2+\sqrt[3]{1-8x^4}}  {1-3\sqrt[5]{x^6}} .$$
Получаем неопределённость: $$\biggl[\dfrac{-\infty}{-\infty}\biggr].$$
Делим на $x^{\frac{4}{3}}$ числитель и знаменатель:
$$\lim\limits_{x\rightarrow+\infty} \dfrac{\dfrac{2}{x^\frac{4}{5}}+\biggl(\dfrac{1}{x^\frac{4}{3}}-8\biggl)}{\dfrac{1}{x^\frac{4}{3}}-\dfrac{3}{x^\frac{2}{15}}} = \biggl[\dfrac{-8}{0}\biggl]=-\infty.$$
в)
$$\lim\limits_{x\rightarrow 0} \dfrac{\sqrt[3]{27+x}-\sqrt[3]{27-x}}{x+2\sqrt[3]{x^4}}.$$
Получаем неопределённость: $$\biggl[\dfrac{0}{0}\biggr].$$
Домножим числитель и знаменатель на сопряженные множители:
$$\lim\limits_{x\rightarrow 0} \dfrac{(\sqrt[3]{27+x}-\sqrt[3]{27-x})(\sqrt[3]{(27+x)^2}+\sqrt[3]{(27+x)(27-x)}+\sqrt[3]{(27-x)^2})}{(x+2\sqrt[3]{x^4})(\sqrt[3]{(27+x)^2}+\sqrt[3]{(27+x)(27-x)}+\sqrt[3]{(27-x)^2})}.$$
Упростим выражение:
$$\lim\limits_{x\rightarrow 0} \dfrac{2x}{x(1+2\sqrt[3]{x})(\sqrt[3]{(27+x)^2}+\sqrt[3]{(27^2-x^2)}+\sqrt[3]{(27-x)^2})}.$$
Сократим одинаковый множитель:
$$\lim\limits_{x\rightarrow 0} \dfrac{2}{(1+2\sqrt[3]{x})(\sqrt[3]{(27+x)^2}+\sqrt[3]{(27^2-x^2)}+\sqrt[3]{(27-x)^2})}=\dfrac{2}{27}.$$
г)
 $$\lim\limits_{x\rightarrow 0} \biggl(5-\dfrac{4}{\cos{2x}}\biggl)^{\frac{1}{\sin^2{x}}}.  $$
Подставляя значение, получим:
$$[ 1^{\infty}].$$
Раскрывая данную неопределённость получим:
$$\exp{\lim\limits_{x\rightarrow 0}\frac{1}{\sin^2{x}}\biggl(4-\dfrac{4}{\cos{2x}}\biggl)}.$$
Вычисляим значение степени:
$$\lim\limits_{x\rightarrow 0}4\frac{1}{\sin^2{x}}\biggl(1-\dfrac{1}{\cos{2x}}\biggl).$$
$$ 4\lim\limits_{x\rightarrow 0}\dfrac{\cos^2{x}-\sin^2{x}-\sin^2{x}-\cos^2{x}}{\cos{2x}\sin^2{x}}.$$
$$ 4\lim\limits_{x\rightarrow 0}\dfrac{-2\sin^2{x}}{\cos{2x}\sin^2{x}}.$$
$$ 4\lim\limits_{x\rightarrow 0}\dfrac{-2}{\cos{2x}}=-8.$$
Искомый предел равен:
$$e^{-8}.$$
д)
$$\lim\limits_{x\rightarrow 0} \biggl(\dfrac{\exp{x}-1}{\arctg2x}\bigg)^\frac{6x}{\lg({1+x})}.$$
Заменим множетели на эквивалентные:
$$\lim\limits_{x\rightarrow 0} \biggl(\dfrac{x}{2x}\bigg)^\frac{6x\ln{10}}{x}=\dfrac{1}{2}^{6\ln{10}}.$$
\newpage
е)
$$\lim\limits_{x\rightarrow 1} \dfrac{1+\cos{\pi x}}{\tg^2(\pi x)}.$$
Получаем неопределённость: $$\biggl[\dfrac{0}{0}\biggr].$$
Выполним замену переменных $t = x - 1;\; t \rightarrow 0$: 
$$\lim\limits_{x\rightarrow 0} \dfrac{1+\cos(\pi t+\pi)}{\tg^2(\pi t+\pi  )}.$$
Применив формулы приведения, получим:

$$\cos(\pi t+\pi)=-\cos(\pi t).$$
$$\tg^2(\pi t+ \pi)= \tg^2(\pi t).$$
Заменим выражения на эквивалетные при $t \rightarrow 0$:

$$ 1-\cos(\pi t)  \sim \dfrac{(\pi t)^2}{2}.$$
$$\tg^2(\pi t)  \sim (\pi t)^2.$$
Подставим данные выражения в числитель и знаменатель искомого предела и сократим получившуюся дробь:
$$\lim\limits_{x\rightarrow 0} \dfrac{t^2\pi^2}{2t^2\pi^2}=\dfrac{1}{2}.$$\\
Ответ:а) $\dfrac{5}{8}$; б)$-\infty$; в)$\dfrac{2}{27}$; г)$e^{-8}$; д)$\dfrac{1}{2}^{6\ln{10}}$;
е)$\dfrac{1}{2}.$.
\newpage
\begin{center}
\textbf{Задача № 3.}   
\end{center}
\textbf{Условие:}
а) Показать, что данные функции $f(x)$ и $g(x)$ являются бесконечно малыми или бесконечно большими
при указанном стремлении аргумента. б) Для каждой функции $f(x)$ и $g(x)$ записать главную часть
(эквивалентную ей функцию)  вида $C(x-x_0)^{\alpha}$ при $x\rightarrow x_0$ или $Cx^{\alpha}$
при $x\rightarrow\infty$, указать их порядки малости (роста). в) Сравнить функции $f(x)$ и $g(x)$ при указанном стремлении.$f(x) = \sqrt[3]{x^2+x\sqrt{x}};\; g(x) = \sqrt{x^3+x+1}$.

\textbf{Решение:}\\
а)
$$\lim\limits_{x\rightarrow +\infty} f(x) = \lim\limits_{x+\rightarrow \infty}\sqrt[3]{x^2+x\sqrt{x}}=+\infty .$$ 
Таким образом, $f(x)$ является ББ при данном стремлении.

Аналогично:
$$\lim\limits_{x\rightarrow +\infty} g(x) = \lim\limits_{x\rightarrow +\infty}\sqrt{x^3+x+1}=+\infty.$$
 $g(x)$ является ББ при данном стремлении.
 
б) При $ x\rightarrow\infty$:
$$ f(x)=\sqrt[3]{x^2+x\sqrt{x}}.$$
$$f(x)\sim x^{\dfrac{2}{3}}. $$
Её порядок роста $\alpha=\dfrac{2}{3} $\\
 При $ x\rightarrow\infty$:
$$ g(x) = \sqrt{x^3+x+1} \sim  {x^{\dfrac{3}{2}}}.$$
Её порядок роста $\alpha=\dfrac{3}{2} $\\
в)
$$\lim\limits_{x\rightarrow \infty} \frac{g(x)}{f(x)} = 0.$$ 
Получается, $f(x) = o(g(x))$.
\newpage
\addcontentsline{toc}{section}{Список литературы}
\begin{thebibliography}{99}
\bibitem{book01} Львовский С.М. Набор и вёрстка в системе \LaTeX, 2003 c.
\bibitem{book02} Котельников И.А. \LaTeX \; 2e по-русски, 2004.
\end{thebibliography}
\end{document}